\documentclass[	10pt,a4paper, UTF8]{ctexart}


\begin{document}

\title{\textbf{“团队协作与职业素质”课程报告}}
\author{计自1302班 \space \space 陈文帆 \space \space 201326810202}
\maketitle

其实,第一次选课的时候,我并没有选这门课,当时选的都是一些“专业课”,像“编译原理”,“数据挖掘”等。但是,我去蹭了本课的第一节课。好玩!这是我对本课的第一印象。早上一二节课,再加上前一天晚上睡得迟,按理说会犯困打瞌睡,可是我那天真的是越上越精神,回寝室之后还对刚起床的室友说:这课你不去上,真是可惜了。然后我开始忧心忡忡了,这课这么好玩,要是没人退课怎么办?索性把心一横,选不上也去听,反正第一节课也是蹭课的。而结果是,我选上了本课,却把其他的“专业课”都给退了。因为一周下来,我感觉本课给我的帮助可能会大于那些“专业课”。而几周的课下来,也证明了我的感觉没错。

我的感觉,老师的授课计划与课名“团队协作与职业素质”如出一辙,也是循着“团队协作”与“职业素质”的顺序。

第一节课,我们组队,做游戏,“开公司”。虽然开始并不知道这个游戏的深意,但小组的每个同学都很投入,合作得也很不错。然后,我特别喜欢我们的“公司”。公司很健全,根据老师介绍的一个公司应具有的基本部门,每个人都选择了自己的部门与职位。而我最喜欢的则是我们公司的名称了,“逸加”(一家)。这是集体智慧的结晶。也正如公司名所示,我们小组真的就像一家,这一点在之后能很好地看出来。

第二节课,我们还是做游戏,并开始显式地学习团队协作与职业素质的知识。游戏很好玩,而职业素质很重要。课上提到的许多素质,比如“沟通”,之前其实都知道,但随着时间的推移,渐渐融入生活的点滴,或者被遗忘。而在我们即将走出校园的这非常时刻,重新拎一拎,很有帮助。课后,各公司着手“招聘”的准备。关于招聘的各项,小组内部讨论了好久,对公司的重定位,发展方向,人才需求等等。一个同学当时就开玩笑地说:我感觉我们可以去创业了。好吧,这种为公司的发展,出谋划策尽心尽力的感觉真的很棒!

第三节课,居然是一场“面试”。作为应聘者参加完面试,我确实发现了一些不足,在一张小纸上作了记录,可能以后会有帮助,看来“预演”很有必要。之后作为面试官,又有另一番体验。课后的录取讨论,小组内部讨论了许久,让我从一个企业,一个需求者的角度看待问题,对于企业需要什么样的人才,或者需要的人才应具备怎样的素质,有了更深的体会。

第四节课,算是前几课的归纳总结吧。没什么好说。

以上的种种,都是我站在现在这个时间节点上,思虑过往有的体会。有些东西,可能老师的本意并非如此,但所谓“述者无意,听者有心”,以上便是我在课堂上的收获。

在浙大科技园实习双选会之前,在去企业实习之前,我对于本课程为我们提供这样的机会——能够提前进入企业去实习锻炼,都是心怀感激的。而我也认为,这正是本课程的终极意义所在。因此,课内,我很用心地上好每一堂课;课外,我也尽我所能地参加双选会,参加面试,去企业实习。但不知为何,实习双选会后的一切,来得总不尽如人意。经历了这些,我开始明白学校要求各班开展“做情绪的主人”心理班会的原因所在了。

个人认为自己对于双选会还算比较上心的,简历比较早就已经在准备了,还特地找了身为HR的姐姐为我修改。双选会前一天,老师上传了来参加实习双选会的企业的资料,那天晚上我对着公司简介与岗位需求研究了很久,将资料仔细看了一遍,也上网看了许多公司的介绍,对着原文件删减了2遍,又精简了1遍,列了一个重点企业名单。

双选会当天,我总共准备了5份简历,根据各公司的宣讲又重新梳理了重点企业与岗位名单,总共投出3份简历:

\begin{enumerate}
\item 杭州造风电子商务有限公司 \space 应聘岗位:前端开发
\item 杭州云秒科技有限公司    \space \space \space \space \space \space \space 应聘岗位:大数据开发/前端开发
\item 杭州妞诺科技有限公司    \space \space \space \space \space \space \space 应聘岗位:服务器开发
\end{enumerate}

为什么是这3家企业的这几个岗位呢?因为,几乎没有一家企业招“python开发”(我目前最想从事工作),我只好退而求其“次”,应聘\textbf{感兴趣}的“前端开发”。而这可能就是我之后悲剧故事的开端——我并没有应聘\textbf{\textit{能力范围内的}}岗位,比如java开发或者硬件工程师,而是选择了应聘\textbf{\textit{能力范围外的感兴趣的}}岗位。我错误地认为:实习期间,不会的都可以学,因此我希望找一个兴趣相关的岗位。这话没错,但缺少一个大前提,就是“应聘成功”,否则又何来实习一说?空有一腔热血的我,碰壁是在所难免的。至于应聘“大数据开发”,是因为我嗅到了一股“python”的气息,因此申请了这个岗位;“服务器开发”,则是反观自身,发现我现有的知识技能与这个岗位最是契合,基本上满足所有要求,最具竞争力,而我本人也有意向从事相关的工作。

目前看来:前期的简历编写阶段,我准备得较为充分;后期的企业筛选,简历投递,除了一手“服务器开发”尚可,其他的尽是昏招烂招。这可能与我“随性”的性格有很大关系,现在思来,最要不得。

投出去3份简历,实际让我去面试的企业只有2家。据我所知,所有“妞诺”的简历,都石沉大海了。于我来说,这可谓是屋漏偏逢连夜雨。

我第一家去面试的公司的是“造风”。电话约的面试,结果我把时间给弄错了,提前了一天过去,而他们公司的技术人员正好不在。最后跟一位HR模样的男生聊了一些基本情况。从我们的谈话中,我能感受到,他们公司需要的并不是实习生,而是有工作经验的能马上做出业绩的人。而我这种半吊子的前端开发,显然不适合他们公司。事后我想,\textbf{毕竟是一个初创企业,对人才的需求可能会比较高}。这也让我意识到——我全凭兴趣地选择前端开发——是多么错误的决策。临别前,男HR告诉我,第二天他们的技术人员会再与我通过电话进行技术面。可是直到现在也没有音信。这更打击了我——术业有专攻,可能现阶段的我选择前端开发真的不是一个好决定。

据第二天去参加面试的同学反映,那天他们的技术人员还是不在。联想到我去面试的时候,女HR让我填了一张超详尽的信息表(甚至包括婚否,生育情况),还告诉我必须填完整,在我随男HR去面试时,她对着信息表在“拉勾网”上注册信息。我开始怀疑他们公司招人的诚意,是真招人还是仅仅为了提高在“拉勾网”上的关注度?我浏览了一下“拉勾网”上其他用户的面试反馈,很多都表示没有“技术面”。

我第二家去面试的企业是“杭州云秒科技有限公司”。也正是“云秒”,让我觉得本课程后期的工作做得不够好。让我从头开始说起。

双选会当天,“云秒”的胡将和“造风”的郭洋是作为企业代表发言的。胡先生在宣讲的环节大谈梦想,并大肆鼓吹他们比腾讯早了多少年推出了某某产品(说他是鼓吹,是我后来才发现的)。当时浙大科技园方面提供了一个学生与企业互动的微信群聊。在群聊里,胡先生2次共加增了10余个岗位,这让我不禁起了疑惑。当时我就跟同学说了,这个企业有问题,哪里会一次性招收这么多岗位,有这么缺人吗?但我也没有深思,只是看到他们似乎提供“python相关”的岗位,我就投了简历。

面试当天,我提前去了浙大科技园,但是我在所有大楼都找不到“杭州云秒科技有限公司”。在我不知所措的时候,他们的“HR”恰好驱车赶到,解了我的燃眉之急。她领着我直接去了“他们公司”的会议室,因此我没来得及打量他们的“公司”。面试的时候,胡先生和HR都在场,还是没有技术人员。胡先生第一个问题就让我吃了一惊,“你有兴趣做项目经理吗?”我当时就懵了,我一个实习生,就做项目经理了?之后他们关注的重点始终是我一个星期的能有几天到位,依旧让我很纳闷。在我搬出老师与浙大科技园协商的2个星期实习之后,才有所松动。由于是实习,我没有主动提及薪水的问题。他们几乎问了所有问题,也独独没有提及薪水的问题。事后想想,这也是个问题。然后他们就告诉我可以上岗了,我以为找到了实习岗位,开心地直接离开了,又忘了打量他们的“公司”。

我去“云秒”实习的第一天,也是唯一一天。直到那时,我才注意到公司原来是浙大科技园的e-works实验室里的两张办公桌!全公司除了我,没有其他人员了!我终于知道,为什么无论是百度,搜狗,必应还是谷歌,所有的搜索引擎都搜索不到任何与“杭州云秒科技有限公司”相关的信息,连浙大科技园内也没有“杭州云秒科技有限公司”的任何信息。

那天,胡先生让我做公司的主页。我跟他说,我只会做静态网页。他说行。然后我就吭哧吭哧地开始做,模板找好了,还需要放在首页展示的公司信息,我就向胡先生要。他给我的文档里就这么一句话:

公司致力于移动互联网、下一代物联网、云计算和大数据分析开发平台建设。

其他的信息让我自己上网去找。我当时真是无话可说。我又指着模板跟他说,这一块应该放公司的产品。他就跟我说了两个概念“亲情呼叫”和“智慧旅游”。文字还是让我自己去找。我还是无话可说。我指着另一块跟他说,这里应该放我们公司的工作环境。他让我上网找一些浙大科技园的图,如果需要自己再拍几张凑一下。我依旧无话可说。

于是那一整天,我做的事情就是,找模板,修改模板,找图片,去水印,拍照片,找文字,编辑文字。另外我还自己掏钱租了个服务器。蔡老师跟我说过,去初创企业实习可能需要做的事情会比较多,学到的知识技能也会比较多,我当时回复她说,我其实不怕辛苦。但也不应该是这个样子吧?

下班之后,我越想越觉得这个公司有问题,它的业务覆盖了时下最前沿最热门的技术(移动互联网,物联网,云计算,大数据),这几乎算是最顶尖的互联网公司了,却连一份真实的资料都拿不出手。在跟同学跟父母跟辅导员聊过之后,我辞职了。

我的实习,一天就结束了。

我不知道为什么“云秒”这样一个比起“公司”更像“骗子”的企业能作为企业方来参加双选会,甚至它的所谓“CEO”还能作为企业代表发言。我知道企业良莠不齐是肯定的,可我猜到了开头,却猜不中这样的结局。

老师尽心尽力地为我们提供这样一个实习锻炼的机会,我真的很感激。但是不是应该先对企业把把关?像“云秒”这样的“企业”,我觉得坚决不能放它们进来。我不知道我是不是第一个遇到这种情况的学生,但我希望,我是最后一个。

经历了这么多,我暂时也不想找实习工作了,准备这段时间好好充实下自己,静待“5.21实习双选会”的到来。

\end{document}
